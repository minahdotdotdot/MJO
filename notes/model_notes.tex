\documentclass[10pt]{article}
\usepackage{amsmath,amssymb,amsfonts,amsthm,bm}
\usepackage[margin=1in]{geometry}
\usepackage{graphicx}
%\usepackage{hyperref}
\usepackage{xcolor}
\usepackage[sort&compress, numbers]{natbib}
%\usepackage{ulem}

\newcommand{\dd}[3]{\frac{\text{d}^{#3}{#1}}{\text{d}{#2}^{#3}}}
\newcommand{\pd}[1]{\partial_{#1}}
\newcommand{\ov}[1]{\overline{#1}}
\newcommand{\bu}{\bm{u}}
\newcommand{\bbu}{\ov{\bu}}
\newcommand{\rmd}{\,\mathrm{d}}
\newcommand{\rmD}{\,\mathrm{D}}
\newcommand{\HH}{\mathcal{H}}
\renewcommand{\arraystretch}{2}
\title{Dimensional Basic Models and Parameters}
\author{Minah Yang}
\date{\today}
\begin{document}
\maketitle
\section{Basic Models}
\subsection{Notation}
Define the following operators for some quantity, $a$.
\begin{align*}
\frac{\rmd}{\rmd t} a &= \partial_t a + \nabla \cdot \left(\vec{u}a\right) + \partial_z (wa)\\
\Delta a &= \partial_t a + \nabla \cdot \left(\vec{u}a\right)
\end{align*}
where $\vec{u} = (u,v)$, just the horizontal components of velocity.
Note that $\frac{\rmd}{\rmd t}$ is equivalent to the standard 3-D Lagrangian derivative as a result of  incompressibility: 

\begin{equation}
\nabla \cdot \vec{u} +\partial_z w= 0
\label{incoma}
\end{equation}

\begin{align*}
\frac{\rmd}{\rmd t} a &= \partial_t a + \nabla \cdot \left(\vec{u}a\right) + \partial_z (wa)\\
&= \partial_t a + (\nabla \cdot \vec{u}) a + \vec{u} \cdot \nabla a + (\partial_z w)a + w \partial_z a  \text{,\quad (Product Rule)}\\
&= (\partial_t  +\vec{u} \cdot \nabla  +w \partial_z) a
\end{align*}
\subsection{Single-Layer Models}
I looked at the Bouchut et al (2009) paper for the single-layer equations.
\begin{align}
\partial_t \vec{v} + \left( \vec{v}\cdot \nabla \right)\vec{v} &= -g\nabla h - f\hat{z}\times \vec{v}\\
\partial_t h + \nabla \cdot \left(\vec{v}h\right) &= -\gamma P \\%= -\frac{(h-h_0)}{\tau}H(h_0-h)\\
\partial_t Q + \nabla \cdot \left(\vec{v}Q\right) &= -P. \label{eq3}
\end{align}
The BLLZ09 model doesn't have an interesting climatology because there is no moisture source; either the water will all rain out or $Q$ will become so small that there is no longer any precipitation to drive the dry dynamics.
(The $-\gamma P$ term is a model of convective heating, not radiative cooling.) 

Craig and Mack suggest a model where the change in moisture (CWV) with respect to time is dependent on three physical phenomena: subsidence drying, convective moistening, and horizontal transport
\[\partial_t Q + \nabla \cdot \left(\vec{v}Q\right) = S + C + T.\]
Their `subsidence drying' $S$ doesn't really fit within the BLLZ09 framework, and we already have horizontal transport via the $\nabla\cdot(\vec{v}Q)$ term.
CM13's `convective moistening' is a combination of both moistening and precipitation.
Specifically their total moistening plus precipitation term has the form
\[C = \frac{1-\epsilon}{\epsilon} P = -P +\frac{1}{\epsilon}P\]
where $P/\epsilon$ is the moistening term and $\epsilon$ is a `precipitation efficiency.'
They use 
\[\epsilon = \hat{\beta}Q/Q_s\]
where $Q_s$ is a saturation level and $\hat{\beta} = 1.1$.
We will add the CM13 moistening model to our equations so that the water doesn't all rain out.\\

Another drawback to the BLLZ09 model (in terms of climatology) is that the $h$ tendency is uniformly non-positive (because precipitation is non-negative $P\ge0$).
If we add a moistening term so that $P$ doesn't eventually go to zero, the $h$ can only decrease.
In reality there should be a balance between convective heating and radiative cooling (radiative-convective equilibrium; RCE).
To enable this kind of balance we need to add radiative cooling to the one-layer model.
A very simple way of adding this is as a linear restoring that is only active when $h$ falls below some threshold $h_0$ (the on/off behavior means that the term isn't completely linear).

Adding convective moistening and radiative cooling to the BLLZ09 model results in the following
\begin{align}
\partial_t \vec{v} + \left( \vec{v}\cdot \nabla \right)\vec{v} &= -g\nabla h - f\hat{z}\times \vec{v}\\
\partial_t h + \nabla \cdot \left(\vec{v}h\right) &= -\gamma P -\frac{(h-h_0)}{T_{RC}}H(h_0-h)\\
\partial_t Q + \nabla \cdot \left(\vec{v}Q\right) &= -P + \frac{1}{\epsilon}P.
\end{align}

Hottovy \& Stechmann showed that a very simple moisture/water equation can reproduce the observed background variability.
Their model included only diffusion, moistening, and precipitation, but they needed a stochastic term to represent fine-scale moistening or convective processes.
We will also add a similar term, but we want to avoid having the stochastic forcing generate negative $Q$ values, so we will make it multiplicative (cf geometric brownian motion).
Our upgraded BLLZ09 model is therefore

\begin{align}
\partial_t \vec{v} + \left( \vec{v}\cdot \nabla \right)\vec{v} &= -g\nabla h - f\hat{z}\times \vec{v}\\
\partial_t h + \nabla \cdot \left(\vec{v}h\right) &= -\gamma P -\frac{(h-h_0)}{T_{RC}}H(h_0-h)\\
\partial_t Q + \nabla \cdot \left(\vec{v}Q\right) &= -P + \frac{1}{\epsilon}P + \sigma_Q \tanh(3Q/Q_s)\dot{W}_Q.
\end{align}
The term $\dot{W}_Q$ represents Gaussian white noise in time and space. 
We should really assume that this is Stratonovich rather than Ito; on the other hand we're not trying to make a careful model of a physical process, so it's fine to just assume it's Ito.
%Also we might want to replace the white noise by red noise (time-correlated) with a non-Gaussian distribution following the work of Penland and Sardeshmukh; this wouldn't change the spectrum in Hottovy \& Stechmann's linear model, but it could definitely change our behavior. Actually not sure the Penland & Sardeshmukh model is useful here.

\subsection{Two-layer, $Q_2 = 0$, no subsidence drying}


We will represent convective moistening (and drying) with two mechanisms: precipitation and moistening from detrainment.
Precipitation, $p$, removes moisture as water vapors condense to water droplets and drop to the surface.
Moistening via detrainment, $m$, happens at the top of clouds, and we assume its relationship to precipitation as in Equation \ref{eqn:conv_moist}.
Following \citet{CM2013}, we formulate convective moistening with a combination of precipitation and detrainment moistening.
\begin{equation}
c = \frac{1-\epsilon}{\epsilon} p = -p + \frac{1}{\epsilon}p = -p + m
\label{eqn:conv_moist}
\end{equation}
The negative sign on the $p$ is analogous to losing moisture, and moistening proportional to precipitation.
\begin{equation}
m = \frac{1}{\epsilon} p
\end{equation}

Consider an equation for the moisture with the following physical phenomena: convective moistening (precipitation, moistening) and horizontal transport.
\begin{equation}
\partial_t q = -p + m + t
\end{equation}
Horizontal transport, $t$, can be understood as a diffusive term that represents horizontal mixing.
{\color{blue} We have:
\begin{equation}
\Delta Q = -P + M
\label{neweq}
\end{equation} 
}
In addition, we include radiative cooling as a moisture source through conservation of moist enthalpy.  
Radiative cooling is related to the temperature, $\theta$.
\begin{equation}
\frac{\rmd}{\rmd t}\theta = rc
\end{equation}

We rely on an analysis of the two-layer shallow water equations coupled with moisture shown in \citet{LLZB2011}.
Here, conservation of moist enthalpy is assumed to be achieved when precipitation is the only moisture sink. 
Since our model includes radiative cooling as well as moistening through detrainment, we adjust the analysis starting from the conservation of moist enthalpy. 
Let $me$ represent moist enthalpy, and $rc$, radiative cooling.

Conservation of moist enthalpy is achieved when radiative cooling and moistening were excluded. 
We use this conservation law to extract how moist enthalpy evolves.

\begin{align*}
\frac{\rmd}{\rmd t} me = \frac{\rmd}{\rmd t}\left(\theta + \frac{L}{c_p}q\right) &= \frac{\rmd}{\rmd t}\theta + \frac{L}{c_p}\left(\partial_t q + \vec{u}\cdot \nabla q + w\partial_z q\right) \\
&= \frac{\rmd}{\rmd t}\theta + \frac{L}{c_p}\left(-p + m + t+ \vec{u}\cdot \nabla Q + w\partial_z Q\right) \\
&= rc +\frac{L}{c_p} m \\
&= rc +\frac{L}{c_p} \left(\frac{1}{\epsilon} p\right) \\
\end{align*}

We compute: $\frac{\rmd}{\rmd t} ME $, and add in our results from incompressibility and conservation of mass.
\begin{align*}
\partial_t h +\vec{u}\cdot \nabla h &= \partial_t h +\nabla(\vec{u}\cdot h) + w(h) \text{\quad (Product Rule and incompressibility)}\\
&= \Delta h + w(h) = -W + w(h)\text{\quad (Conservation of Mass) }
\end{align*}
\begin{equation}
\Delta ME = \Delta \int_{0}^{h} me \rmd z= me(h)\left( -W + w(h) \right) + \int_{0}^{h} \partial_tme \rmd z + \left(\nabla \cdot\vec{u} \right)\int_{0}^{h} me \rmd z + \vec{u}\cdot \int_{0}^{h}\nabla me \rmd z 
 \label{eqdt}
\end{equation}
  
Although we no longer have conservation of moist enthalpy, we can still compute $\frac{\rmd}{\rmd t} me$, and vertically integrate it to get another expression for the integral terms in Equation \ref{eqdt}.  
\begin{align}
\int_{0}^{h} \frac{\rmd}{\rmd t} me \rmd z &= \int_{0}^{h} \partial_t me \rmd z + \int_{0}^{h} (\nabla \cdot \vec{u}) me \rmd z + \int_{0}^{h} \vec{u} \cdot \nabla me \rmd z + \int_{0}^{h} \partial_z (w me) \rmd z \\
&= \int_{0}^{h} \partial_t me \rmd z + (\nabla \cdot \vec{u})\int_{0}^{h}  me \rmd z + \vec{u} \cdot \int_{0}^{h} \nabla me \rmd z + w(h)me(h) \\
&= RC + \frac{L}{c_p} M \label{eq4}
\end{align}
{\bf Assuming $RC = \int_0^h rc$ and similar for $M$?}
Combining \ref{eqdt} and \ref{eq4} results in:

\begin{equation}
 \Delta ME = -W me(h) + RC + \frac{L}{c_p} M
\end{equation}

We deconstruct $ME$ into its components to get another relationship between $P$, $W$, and $RC$. 
{\bf The above derivation doesn't really mention the fact that we have two layers\ldots}
\begin{align*}
\Delta \left(\int_{0}^{h}\theta \rmd z + \frac{L}{c_p}Q \right) &= -W me(h) + RC + \frac{L}{c_p} M \\
\theta_1(-W) + \frac{L}{c_p}\Delta Q &=  -W me(h) + RC + \frac{L}{c_p} M \\
W \left(me(h) - \theta_1 \right) &= RC + \frac{L}{c_p} M  - \frac{L}{c_p}\Delta Q\\
\Delta Q &=\frac{c_p}{L} \left(W \left(\theta_1 - me(h) \right) + RC + \frac{L}{c_p} M  \right)
\end{align*}
{\bf Although Lambaerts et al.~use the symbol $\beta$ here, we shouldn't because the symbol universally means $\beta = (2\Omega/R_E)$ where $\Omega$ is the rotation rate of the Earth (radians per time) and $R_E$ is the radius of the Earth.}

Fitting this with our assumption from before (\ref{neweq}) and the choice of a ``dry'' stable stratification of the atmosphere we result in:

\begin{align}
W &= \beta\left(P+\frac{c_p}{L} RC\right)\\
\beta &= \frac{L}{c_p}\frac{1}{\theta_2-\theta_1}
\end{align}


\section{Dimensional Governing Equations}
Velocity Equations:
\begin{align}
\partial_t \vec{u_1} + (\vec{u_1}\cdot \nabla )\vec{u_1} +  f\hat{k} \times \vec{u_1} &= -g \nabla (h_1 + h_2)	\label{eqn:DVel}\\
\partial_t \vec{u_2} + (\vec{u_2}\cdot \nabla )\vec{u_2} +  f\hat{k} \times \vec{u_2} &= -g \nabla (h_1 + \alpha h_2)	 + \frac{\vec{u_1}-\vec{u_2}}{h_2}\beta \left(P + \frac{c_p}{L}RC \right)
\end{align}
Here, $\alpha = \frac{\theta_2}{\theta_1}$. 
We also assume that $W_2 \equiv 0$.

Conservation of Mass (height):

\begin{align}
\Delta h_1 &= -W_1 = -\beta \left(P + \frac{c_p}{L}RC\right)\\
\Delta h_2 &= -W_2 + W_1 = \beta \left(P + \frac{c_p}{L}RC\right)
\label{eqn:DHei}
\end{align}

Moisture Dynamics:

\begin{align}
\Delta Q &= -P + M\\
&= \left(-1 + \frac{1}{\epsilon} \right)P+ \text{ noise}
\label{eqn:DMoi}
\end{align}

\subsection{Parameter Values}
The standard units we use are seconds, meters, and kilograms. Note that for water:
\begin{align*}
\frac{kg}{m^2} &= \frac{L}{m^2} = \frac{1000 mL}{m^2}\\
&= \frac{1000 cm^3}{ (100cm)^2} = \frac{1}{10} cm = 1 mm = 0.001 m 
\end{align*}

We also approximate $\alpha = \frac{\theta_2}{\theta_1}$ via the relation $\sqrt{g'H} = \sqrt{g(\alpha -1) H} \approx 30 $m$s^{-1}$ (speed of Kelvin wave).

This yields $\alpha \approx 1+\frac{900}{5000g} \approx 1.02$.


\begin{center}
	\begin{tabular}{||c |c|c|| } 
		\hline
		fixed & $g$  & $9.80665 m/s^2$ \\  \hline
		fixed & radius of earth & $6,371,000$ m \\ \hline
		not fixed & $\alpha$  & $\frac{\theta_2}{\theta_1}\approx 1.02$ \\ \hline
		not fixed & $\beta$ &  $\approx 750$ \\ \hline
	\end{tabular}
\end{center}

\section{Radiative Cooling and Precipitation Models}
In order to ``close'' the governing equations (Eq. \ref{eqn:DVel}--\ref{eqn:DMoi}), we consider several models for $P$ and $RC$.

\subsection{Radiative Cooling}

We formulate the radiative cooling function with the following:
\begin{equation}
\beta \frac{c_p}{L} RC = -\frac{h_2-h_1}{T_{RC}}\HH(h_2-h_1)
\end{equation}
where $\HH$ is the Heaviside function.{\color{blue}$\frac{c_p}{L}$ has the units of temperature.} {\bf to balance the units of $RC$ which are temperature times length per time.} The Heaviside function ensure that only radiative cooling (and not heating!) occurs. 
That is, when the upper layer  is taller ($h_2 > h_1$), the whole term is negative. 
$T_{RC}$ is introduced to scale the cooling phenomena. 
We will approximate it to be $\approx 16$ days. 
{\color{blue} With the nondimensionalized variables, we result in:
	\begin{equation}
	\beta \frac{c_p}{L} RC = -H\frac{h_2-h_1}{T_{RC}}\HH(h_2-h_1).
	\end{equation}
}

\subsection{Betts-Miller}
In this model, rain rate is given by some proportion ($\tau_q$) of the amount of moisture above some saturation level ($Q_s$).
\begin{equation*}
P(Q) = \frac{Q-Q_s}{\tau_q}\HH{(Q-Qs)},
\end{equation*}
where $\HH(\cdot)$ denotes the Heaviside function. 
Hottovy \& Stechmann use $\tau_d=96$ hours, which is extremely slow.
To get rain rates comparable to the (observationally motivated) models in the next two sections requires $\tau_q$ between $1/30$ and $1/10$ hours.
The mismatch between HS15's precipitation parameters and ours will possibly require us to increase the noise forcing amplitude $\sigma_Q$ from $50$ mm / hr$^{1/2}$ to about $50\times\sqrt{20}$ mm / hr$^{1/2}$.
I made this estimate assuming that we are still using HS15's diffusion coefficient $750$ km$^2$ / hr.


\subsection{Craig \& Mack \citep{CM2013}}
The precipitation model is the following:
\begin{equation}
P(Q) = a(t)\left(\exp(b\frac{Q}{Qs})-1\right),
\end{equation}
where 
\begin{equation}
a(t) = \frac{P_{ave}}{\frac{1}{A}\int \left( \exp(b\frac{Q}{Qs})-1 \right)\rmd A}
\end{equation}

This $a(t)$ is used by Craig \& Mack to enforce a constant total amount of precipitation over the area at each time-step.
We don't want to enforce a fixed total precipitation rate, so we will replace the average on the bottom by a constant; for example if we set $Q=Q_s/\hat{\beta}$ then we'd get a constant of about 32,000 in the denominator.
It's unlikely that $Q = Q_s/\hat{\beta}$ throughout the domain; instead it is probably smaller.
So we need to try a range of values in the denominator and can start around 15,000.
The quantity $8$ kg m$^{-2}$ day$^{-1}$ is a reasonable estimate for $P_{ave}$ in radiative-convective equilibrium.
If $Q=Q_s/\hat{\beta}$ this leads to a rain rate of about 0.7 mm/hr, which is quite small.
Recall that kg m$^{-2}$ day$^{-1}$ = mm day$^{-1}$ = $\frac{1}{86,400,000}$ m s$^{-1}$. 
\subsubsection{Parameters}
\begin{center}
	\begin{tabular}{||c |c|c|| } 
		\hline
		not fixed & $\epsilon$ &   $\delta \frac{Q}{Qs}$ \\ \hline
		not fixed & $\delta$ & $ \approx 1.1$  \\ \hline
		not fixed & $b$& $ \approx 11.4$\\ \hline
	\end{tabular}
\end{center}

\subsection{Neelin et al.~2009}
Neelin, Peters, and Hales (JAS 2009) fit precipitation rates to $Q$ from observational data.
The fit a power law to precipitation and end up with
\[P(Q) = a\left(\frac{Q-Q_s}{Q_s}\right)^{1/4}\]
(actually their best-fit exponent was .23; see caption of figure 2).
The best-fit value of $a$ was not reported, but eyeballing it I guess $a\approx 4.4$ mm/hr.
There's clearly nonzero $P$ below the cutoff $Q_s$ in their data.
Rather than try to change their model to account for this, I think using it in a model with memory (see next section) will lead to nonzero $P$ below $Q_s$. 

\subsection{MJO Skeleton}
The MJO skeleton model of Majda \& Stechmann (PNAS 2009) has an unusual model for precipitation.
Their model is linear and involves deviations from a background state, so it's not obvious how to reconstruct a full precipitation model.
In their linear model the precipitation rate is proportional to `wave activity' $a$, and the time-tendency of wave activity is proportional to moisture anomalies.
A simple way to achieve this is to replace $a$ by $P$ (since they are related by a proportionality constant), and to use the model
\[\dot{P} = \frac{P_0-P}{\tau_{MS09}}\]
where $P_0$ is one of our other models for $P$ (e.g. Betts-Miller), and
where $\tau_{MS09}$ is a time scale that measures the rate at which precipitation responds to moisture anomalies.
The above model ensures that $P$ never becomes negative (as long as the initial condition is positive), and adds some memory to the precipitation model.
Appropriate values for $\tau_{MS09}$ are very unclear, but we know that as $\tau_{MS09}\to0$ the model loses its memory and becomes $P=P_0$.
We should probably vary it and see what happens as it becomes larger than or smaller than the other time scales in the system.

The above model should be applied in a Lagrangian sense, yielding
\[\Delta P = \frac{P_0-P}{\tau_{MS09}}.\]
If the velocity was nondivergent this would avoid situations where $P$ is large enough to cause negative $Q$ (at least in the absence of noise and numerical errors).
As it stands, it's not clear if this formulation will avoid negative $Q$.\\

The 2009 paper used a linear formulation that the above is meant to mimic. 
Later papers use a nonlinear version that was mentioned but not used in the 2009 paper.
In the nonlinear model the precipitation tendency is proportional to both $P$ and $Q$ anomalies.
We can model this somewhat crudely by making the restoring time scale $\tau_{MS09}$ related to $Q$.
Specifically, if we let $\tau_{MS09}\to \tau_{MS09}Q_s/Q$ then we obtain the following model
\[\Delta P = \left(\frac{P_0-P}{\tau_{MS09}}\right)\frac{Q}{Q_s}.\]
Qualitatively this means that if $Q$ is close to $Q_s$ then the restoring is on time scale $\tau_{MS09}$.
If $Q\ll Q_s$ then the restoring is very slow (long time scale), whereas if $Q\gg Q_s$ then the restoring is rapid (short time scale).

\subsection{Hottovy and Stechman}
In a 2015 SJAM paper Hottovy \& Stechmann analyze four models for precipitation and moistening at a point (no horizontal extent; `single column').
Precipitation and moistening are modeled as exponential growth and decay, respectively.
Their models switch between precipitation and moistening either deterministically or stochastically.
The simplest model switches deterministically between moistening and precipitation when $Q$ crosses a threshold $Q_c$.
The other deterministic model has hysteresis: the model switches from moistening to precipitation as $Q$ increases past $Q_c$, and it switches from precipitation to moistening as $Q$ decreases past $Q_{np}$.
In the stochastic models the switch is not instantaneous; instead, the time it takes to switch after $Q$ crosses a threshold is an exponentially-distributed random variable.
The models with hysteresis have `memory.'
I'm not sure if we need to try to do something like this; the skeleton model has memory too and is far easier to implement.

\subsection{Stochastic precipitation models with memory}



\clearpage
\bibliographystyle{plainnat}
\bibliography{refs}


\end{document}