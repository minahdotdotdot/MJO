\documentclass[10pt]{article}
\usepackage{amsmath,amssymb,amsfonts,amsthm,bm}
\usepackage[margin=1in]{geometry}
\usepackage{graphicx}
\usepackage{hyperref}
\usepackage{color}
%\usepackage{ulem}

\newcommand{\dd}[3]{\frac{\text{d}^{#3}{#1}}{\text{d}{#2}^{#3}}}
\newcommand{\pd}[1]{\partial_{#1}}
\newcommand{\ov}[1]{\overline{#1}}
\newcommand{\bu}{\bm{u}}
\newcommand{\bbu}{\ov{\bu}}
\newcommand{\rmd}{\,\mathrm{d}}
\newcommand{\rmD}{\,\mathrm{D}}
\newcommand{\HH}{\mathcal{H}}
\renewcommand{\arraystretch}{2}
\title{Dimensional Basic Models and Parameters}
\author{Minah Yang}
\date{\today}
\begin{document}
\maketitle
\section{Basic Models}
\subsection{Single-Layer Model}
I looked at the Bouchut-Laembert paper for the single-layer equations.
\begin{align}
\partial_t \vec{v} + \left( \vec{v}\cdot \nabla \right)\vec{v} &= -g\nabla h - f\hat{z}\times \vec{v}\\
\partial_t h + \nabla \cdot \left(\vec{v}h\right) &= -\gamma P = -\frac{(h-h_0)}{\tau}H(h_0-h)\\
\partial_t Q + \nabla \cdot \left(\vec{v}Q\right) &= ?? \label{eq3}
\end{align}

Craig and Mack suggest a model where the change in moisture (CWV) with respect to time is dependent on three physical phenomena: subsidence drying, convective moistening, and horizontal transport (diffusive?). 

\begin{align}
\partial_t Q + \nabla \cdot \left(\vec{v}Q\right) &= S + C + T\\
S &= -\alpha Q \\
C &= \frac{1-\epsilon}{\epsilon} P \\
T &= K \nabla^2 Q\\
P &= a \left(\exp(b\frac{Q}{Q_s}) -1\right)\\
\epsilon &= \hat{\beta} \frac{Q}{Q_s}
\end{align}
The standard values for some of these parameters are as follows: $\hat{\beta} \approx 1.1$, $b \approx 11.4$. Details on the rest can be found in the Craig and Mack paper.

\begin{itemize}
	\item Radiative cooling appears in this model as $\gamma P$ (layer height equation).
\end{itemize}

\subsection{Two-layer, $Q_2 = 0$, no subsidence drying}
Consider writing the convective moistening term into two parts: precipitation, and the resulting moistening. Let $m$ represent moistening.
We have $c = -p + m$. Following Craig \& Mack's example yields: 
\begin{equation}
c = \frac{1-\epsilon}{\epsilon} p = -p + \frac{1}{\epsilon}p
\end{equation}
The negative sign on the $p$ makes sense since we will be losing mosture. So we get:
\begin{equation}
m = \frac{1}{\epsilon} p
\end{equation}

This is the primary difference between our model and Laembert's model-- precipitation is NOT the only moisture sink. In other words, we have:
\begin{equation}
\Delta Q = -P + M
\label{neweq}
\end{equation}

So, now we consider an equation for the moisture with the following physical phenomena: precipitation, moistening, horizontal transport.
\begin{equation}
\partial_t q = -p + m + t
\end{equation}

We also add that radiative cooling is related to the temperature: 
\begin{equation}
\frac{\rmd}{\rmd t}\theta = rc
\end{equation}

If $rc$ and $m$ were excluded, we would still achieve conservation of moist enthalpy. So we can write:

\begin{align*}
\frac{\rmd}{\rmd t} me = \frac{\rmd}{\rmd t}\left(\theta + \frac{L}{c_p}q\right) &= \frac{\rmd}{\rmd t}\theta + \frac{L}{c_p}\left(\partial_t q + \vec{u}\cdot \nabla q + w\partial_z q\right) \\
&= \frac{\rmd}{\rmd t}\theta + \frac{L}{c_p}\left(-p + m + t+ \vec{u}\cdot \nabla Q + w\partial_z Q\right) \\
&= rc +\frac{L}{c_p} m \\
&= rc +\frac{L}{c_p} \left(\frac{1}{\epsilon} p\right) \\
\end{align*}

We compute: $\frac{\rmd}{\rmd t} ME $, and add in our results from incompressibility and conservation of mass.
\begin{align*}
 \partial_t h +\vec{u}\cdot \nabla h &= \partial_t h +\nabla(\vec{u}\cdot h) + w(h) \text{\quad (Product Rule and incompressibility)}\\
 &= \Delta h + w(h) = -W + w(h)\text{\quad (Conservation of Mass) }
 \end{align*}
 \begin{equation}
\Delta ME = \Delta \int_{0}^{h} me \rmd z= me(h)\left( -W + w(h) \right) + \int_{0}^{h} \partial_tme \rmd z + \left(\nabla \cdot\vec{u} \right)\int_{0}^{h} me \rmd z + \vec{u}\cdot \int_{0}^{h}\nabla me \rmd z 
 \label{eqdt}
\end{equation}
  
Although we no longer have conservation of moist enthalpy, we can still compute $\frac{\rmd}{\rmd t} me$, and vertically integrate it to get another expression for the integral terms in \ref{eqdt}.  
\begin{align}
\int_{0}^{h} \frac{\rmd}{\rmd t} me \rmd z &= \int_{0}^{h} \partial_t me \rmd z + \int_{0}^{h} (\nabla \cdot \vec{u}) me \rmd z + \int_{0}^{h} \vec{u} \cdot \nabla me \rmd z + \int_{0}^{h} \partial_z (w me) \rmd z \\
&= \int_{0}^{h} \partial_t me \rmd z + (\nabla \cdot \vec{u})\int_{0}^{h}  me \rmd z + \vec{u} \cdot \int_{0}^{h} \nabla me \rmd z + w(h)me(h) \\
&= RC + \frac{L}{c_p} M \label{eq4}
\end{align}

Combining \ref{eqdt} and \ref{eq4} results in:

\begin{equation}
 \Delta ME = -W me(h) + RC + \frac{L}{c_p} M
\end{equation}

We deconstruct $ME$ into its components to get another relationship between $P$, $W$, and $RC$. 
\begin{align*}
\Delta \left(\int_{0}^{h}\theta \rmd z + \frac{L}{c_p}Q \right) &= -W me(h) + RC + \frac{L}{c_p} M \\
\theta_1(-W) + \frac{L}{c_p}\Delta Q &=  -W me(h) + RC + \frac{L}{c_p} M \\
W \left(me(h) - \theta_1 \right) &= RC + \frac{L}{c_p} M  - \frac{L}{c_p}\Delta Q\\
\Delta Q &=\frac{c_p}{L} \left(W \left(\theta_1 - me(h) \right) + RC + \frac{L}{c_p} M  \right)
\end{align*}

Fitting this with our assumption from before (\ref{neweq}) and the choice of a ``dry'' stable stratification of the atmosphere we result in:

\begin{align}
W &= \beta\left(P+\frac{c_p}{L} RC\right)\\
\beta &= \frac{L}{c_p}\frac{1}{\theta_2-\theta_1}
\end{align}

\subsection{Radiative Cooling}

We formulate the radiative cooling function with the following:
\begin{equation}
\beta \frac{c_p}{L} RC = -\frac{h_2-h_1}{T_{RC}}\HH(h_2-h_1)
\end{equation}
where $H$ is the heaviside function. ??$\frac{c_p}{L}$ has the units of temperature. ?? The heaviside function ensure that only radiative cooling (and not heating!) occurs. That is, when the upper layer  is taller ($h_2 > h_1$), the whole term is negative. $T_{RC}$ is introduced to scale the cooling phenomena. We will approximate it to be $\approx 16$ days. With the nondimensionalized variables, we result in:
\begin{equation}
\beta \frac{c_p}{L} RC = -H\frac{h_2-h_1}{T_{RC}}\HH(h_2-h_1).
\end{equation}

\section{Dimensional Governing Equations}
Velocity Equations:
\begin{align}
\partial_t \vec{u_1} + (\vec{u_1}\cdot \nabla )\vec{u_1} +  f\hat{k} \times \vec{u_1} &= -g \nabla (h_1 + h_2)	\\
\partial_t \vec{u_2} + (\vec{u_2}\cdot \nabla )\vec{u_2} +  f\hat{k} \times \vec{u_2} &= -g \nabla (h_1 + \alpha h_2)	 + \frac{\vec{u_1}-\vec{u_2}}{h_2}\beta \left(P + \frac{c_p}{L}RC \right)
\label{DVel}
\end{align}
Here, $\alpha = \frac{\theta_2}{\theta_1}$. 
We also assume that $W_2 \equiv 0$.

Conservation of Mass (height):

\begin{align}
\Delta h_1 &= -W_1 = -\beta \left(P + \frac{c_p}{L}RC\right)\\
\Delta h_2 &= -W_2 + W_1 = \beta \left(P + \frac{c_p}{L}RC\right)
\label{DHei}
\end{align}

Moisture Dynamics:

\begin{align}
\Delta Q &= -P + M\\
&= \left(-1 + \frac{1}{\epsilon} \right)P
\label{DMoi}
\end{align}

\subsection{Parameter Values}
The standard units we use are seconds, meters, and kilograms. Note that for water:
\begin{align*}
\frac{kg}{m^2} &= \frac{L}{m^2} = \frac{1000 mL}{m^2}\\
&= \frac{1000 cm^3}{ (100cm)^2} = \frac{1}{10} cm = 1 mm = 0.001 m 
\end{align*}

We also approximate $\alpha = \frac{\theta_2}{\theta_1}$ via the relation $\sqrt{g'H} = \sqrt{g(\alpha -1) H} \approx 30 $m$s^{-1}$ (speed of Kelvin wave).

This yields $\alpha \approx 1+\frac{900}{5000g} \approx 1.02$.


\begin{center}
	\begin{tabular}{||c |c|c|| } 
		\hline
		fixed & $g$  & $9.80665 m/s^2$ \\  \hline
		fixed & radius of earth & $6,371,000$ m \\ \hline
		not fixed & $\alpha$  & $\frac{\theta_2}{\theta_1}\approx 1.02$ \\ \hline
		not fixed & $\beta$ &  $\approx 750$ \\ \hline
	\end{tabular}
\end{center}

\section{Precipitation Models}
\subsection{Betts-Miller}
\subsection{Craig \& Mack}
\subsubsection{Parameters}
\begin{center}
	\begin{tabular}{||c |c|c|| } 
		\hline
		not fixed & $\epsilon$ &   $\delta \frac{Q}{Qs}$ \\ \hline
		not fixed & $\delta$ & $ \approx 1.1$  \\ \hline
		not fixed & $b$& $ \approx 11.4$\\ \hline
		
	\end{tabular}
\end{center}


\end{document}