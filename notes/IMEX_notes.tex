\documentclass[10pt]{article}
\usepackage{amsmath,amssymb,amsfonts,amsthm,bm}
\usepackage[margin=1in]{geometry}
\usepackage{graphicx}
\usepackage{hyperref}
\usepackage{xcolor}
\usepackage[linesnumbered,ruled,vlined]{algorithm2e}
%\usepackage{ulem}

\newcommand{\mycommfont}[1]{\ttfamily\textcolor{blue}{#1}}
\SetCommentSty{mycommfont}
\newcommand{\dd}[3]{\frac{\text{d}^{#3}{#1}}{\text{d}{#2}^{#3}}}
\newcommand{\pd}[1]{\partial_{#1}}
\newcommand{\mbfb}[1]{{\color{blue}\mathbf{#1}}}
\newcommand{\mbfr}[1]{{\color{red}\mathbf{#1}}}
\newcommand{\mbfg}[1]{{\color{olive}\mathbf{#1}}}
\newcommand{\ov}[1]{\overline{#1}}
\newcommand{\bu}{\bm{u}}
\newcommand{\bbu}{\ov{\bu}}
\newcommand{\rmd}{\,\mathrm{d}}
\newcommand{\rmD}{\,\mathrm{D}}
\newcommand{\HH}{\mathcal{H}}
\renewcommand{\arraystretch}{2}
\title{Semi-Implicit Time Integration}
\author{Minah Yang}
\date{\today}
\begin{document}
\maketitle
\section{Governing Equations}
We modify the original set of governing equations by setting $h_i=1+\eta_i$, and separating linear and non-linear terms. We will be treating linear terms implicitly, and non-linear terms explicitly. 
\subsubsection*{Momentum}
\begin{align*}
\pd{t}\vec{m_1} =& - \frac{1}{Ro}\left(\hat{k}\times \vec{m_1}\right)
-\frac{1}{Fr^2}\nabla\left(\eta_1+\eta_2\right)+ \kappa \nabla ^2 \vec{m_1}\quad\text{(Linear)}\\
&- \nabla \cdot \left(\frac{1}{h_1}\vec{m_1}\vec{m_1}\right) -\frac{1}{Fr^2} \eta_1\nabla\left(\eta_1 + \eta_2\right) -\frac{\vec{m_1}}{1+\eta_1} \left( \beta \frac{\hat{Q}}{H}\hat{P}(Q) - \frac{T}{T_{RC}}(\eta_2-\eta_1)\HH{(\eta_2-\eta_1)}\right) \quad\text{(Nonlinear)}
\end{align*}

\begin{align*}
\pd{t}\vec{m_2} =& - \frac{1}{Ro}\left(\hat{k}\times \vec{m_2}\right)
-\frac{1}{Fr^2}\nabla\left(\eta_1+\alpha \eta_2\right)+ \kappa \nabla ^2 \vec{m_2}\quad\text{(Linear)}\\
&- \nabla \cdot \left(\frac{1}{1+\eta_2}\vec{m_2}\vec{m_2}\right) -\frac{1}{Fr^2} \eta_1\nabla\left(\eta_1 + \alpha\eta_2\right) -\frac{\vec{m_2}}{1+\eta_2} \left( \beta \frac{\hat{Q}}{H}\hat{P}(Q) - \frac{T}{T_{RC}}(\eta_2-\eta_1)\HH{(\eta_2-\eta_1)}\right) \quad\text{(Nonlinear)}
\end{align*}


\subsubsection*{Height/Mass}

\begin{align*}
\pd{t}\eta_1 =& -\nabla \cdot\vec{m_1} + \kappa \nabla ^2 \eta_1  \quad\text{(Linear)} \\ &-\left(\beta\frac{\hat{Q}}{H}\hat{P}(Q)-\frac{T}{T_{RC}}(\eta_2-\eta_1)\HH(\eta_2-\eta_1)\right)\quad\text{(Nonlinear)}
\end{align*}
\begin{align*}
\pd{t}\eta_1 =& -\nabla \cdot\vec{m_2} + \kappa \nabla ^2 \eta_2  \quad\text{(Linear)} \\ &+\left(\beta\frac{\hat{Q}}{H}\hat{P}(Q)-\frac{T}{T_{RC}}(\eta_2-\eta_1)\HH(\eta_2-\eta_1)\right)\quad\text{(Nonlinear)}
\end{align*}

\subsubsection*{Moisture}
\begin{align*}
\pd{t} Q =& \kappa \nabla ^2 q \quad\text{(Linear)}\\
& + \nabla \cdot (\vec{u_1}Q) = \left(-1+\frac{1}{\epsilon}\right) \hat{P}(Q) \quad\text{(Nonlinear)}
\end{align*}

\subsection{Basic Scheme}
Consider a set of ODE's with linear $L(x)$  and nonlinear $N(x)$ terms.
\begin{equation}
\pd{t}(
\vec{x}) = L(\vec{x})+N(\vec{x}) 
\end{equation}
The discretization is given by treating the nonlinear, explicit terms as the forcing terms in the linear, implicit scheme. Below is an example using the forward Euler time-stepping method. 
\begin{align*}
\frac{1}{h}\left(\vec{x}^{(n+1)}-\vec{x}^{(n)}\right) &= L(\vec{x}^{(n+1)}) + N(\vec{x}^{(n+1)}) \\
\vec{x}^{(n+1)} &= \vec{x}^{(n)}+h\left(L(\vec{x}^{(n+1)}) + N(\vec{x}^{(n)}) \right)
\end{align*}

Solving for $\vec{x}^{(n+1)}$ includes the time integration step as well. 

\subsection{Linear Operator}
We collect the linear terms and try to simplify the matrix linear operator. For now, we exclude the diffusion terms that were previously used to improve numerical stability, which eliminates linear terms in the $q$ equation.

\[\pd{t}\vec{x} = L\vec{x}=
\left[\begin{array}{ccc|ccc}
0 & \frac{1}{Ro}y & -\frac{1}{Fr^2}\pd{x} & 0 & 0 & -\frac{1}{Fr^2}\pd{x} \\
-\frac{1}{Ro}y & 0 & -\frac{1}{Fr^2}\pd{y} & 0 & 0 & -\frac{1}{Fr^2}\pd{y}\\
-\pd{x} & -\pd{y} & 0 & 0 & 0 & 0 \\ \hline
0 & 0 & -\frac{1}{Fr}\pd{x} & 0 & \frac{1}{Ro}y &  -\frac{\alpha}{Fr^2}\pd{x} \\
0 & 0 & -\frac{1}{Fr}\pd{y} & -\frac{1}{Ro}y  & 0&  -\frac{\alpha}{Fr^2}\pd{y} \\
0 & 0 & 0 & -\pd{x} & -\pd{y} & 0
\end{array}\right] \begin{bmatrix}
m_1\\
n_1\\
\eta_1\\
m_2\\
n_2\\
\eta_2
\end{bmatrix} = \begin{bmatrix}
\pd{t}m_1\\
\pd{t}n_1\\
\pd{t}\eta_1\\
\pd{t}m_2\\
\pd{t}n_2\\
\pd{t}\eta_2
\end{bmatrix}
\]


\subsubsection{Option 1: FFT in x-direction } Since our domain has zonal periodicity, we FFT in the $x$-direction. Our original spatial domain from $0^{\circ}$ to $359.75^{\circ}$ longitudes is transformed in the $k-$wave number space. Furthermore, operator $\pd(x)$ is now $ik$. 
Recall $x[1441]=x[1]$. So, our wave numbers should range from $k=1$ to $k=720$. 
\[\pd{t}\vec{x} = \hat{L}_k\vec{x}=
\left[\begin{array}{ccc|ccc}
0 & \frac{1}{Ro}y & -\frac{1}{Fr^2}ki & 0 & 0 & -\frac{1}{Fr^2}ki \\
-\frac{1}{Ro}y & 0 & -\frac{1}{Fr^2}\pd{y} & 0 & 0 & -\frac{1}{Fr^2}\pd{y}\\
-ki & -\pd{y} & 0 & 0 & 0 & 0 \\ \hline
0 & 0 & -\frac{1}{Fr^2}ki & 0 & \frac{1}{Ro}y &  -\frac{\alpha}{Fr^2}ki \\
0 & 0 & -\frac{1}{Fr}\pd{y} & -\frac{1}{Ro}y  & 0&  -\frac{\alpha}{Fr^2}\pd{y} \\
0 & 0 & 0 & -ki & -\pd{y} & 0
\end{array}\right] \begin{bmatrix}
m_1\\
n_1\\
\eta_1\\
m_2\\
n_2\\
\eta_2
\end{bmatrix} = \begin{bmatrix}
\pd{t}m_1\\
\pd{t}n_1\\
\pd{t}\eta_1\\
\pd{t}m_2\\
\pd{t}n_2\\
\pd{t}\eta_2
\end{bmatrix}
\]

Goal: Find diagonalization of $\hat{L}_k V =V D_k \Rightarrow \hat{L}_k = V D_k V^{-1} \Rightarrow D_k = V^{-1}\hat{L}_k V$, such that the eigenvectors are independent from $k$.  

If that's possible, then can solve for $\vec{x}^{(n+1)}$:
\begin{align*}
\vec{x}^{(n+1)} &= \vec{x}^{(n)}+h\left(L_k(\vec{x}^{(n+1)}) + N(\vec{x}^{(n)}) \right) \\
(I-h L_k)\vec{x}^{(n+1)} & = h N(\vec{x}^{(n)}) + \vec{x}^{(n)}\\
V^{-1}(I-h VD_kV^{-1})\vec{x}^{(n+1)} & = V^{-1}RHS\\
(I-h D_k) (V^{-1})\vec{x}^{(n+1)} & = V^{-1}RHS \\
\vec{x}^{(n+1)} & = V (I-h D_k)^{-1} V^{-1}RHS 
\end{align*}

Multiplication of $V$ and $V^{-1}$ can be accelerated using the LU decomposition, and $(I-h D_k)$ is a diagonal matrix, so it is easy to compute the its inverse.

\subsubsection{Option 2: Decouple the two layers by using a transformation of the momenta into barotropic and baroclinic components}

\subsubsection{Option 3: Remove Coriolis terms from linear implicit scheme, and FFT in x-direction and DCT/DST in y-direction}
\begin{itemize}
	\item Justification for removing the Coriolis terms:
	\begin{itemize}
		\item The high-frequency waves in our system are the small-scale gravity waves.
		\item Earth's rotation has little effect on these waves
		\item So, moving the Coriolis terms to the explicit part of the scheme should not have an effect on the high frequency gravity waves.
	\end{itemize}
	\item Benefits for removing the Coriolis terms:
	\begin{itemize}
		\item The removal of Coriolis terms allows us to take a Fourier transform in the y-direction as well.
		\item With the Coriolis terms, we had $\frac{1}{Ro}y\hat{k}\times \vec{m_i}$. Multiplication in physical space becomes a convolution in the Fourier space. This is inconvenient, since this implies that the different modes are depend on each other. 
		\item The new linear system should be simpler.
		\item The total system is now block diagonal. 
	\end{itemize}
	\item Boundary Conditions:
	\begin{itemize}
		\item All variables are zonal-periodic. 
		\item Dirichlet BC: Meridional momenta ($n_i$) are set to zero at the y-boundaries, since we assume that meridional momenta are conserved. Therefore, we use the sine transform. 
		\item Neumann BC: Change in zonal momenta ($m_i$) and height fluctuations ($\eta_i$) across the y-boundaries are zero {\color{blue}(Why?)}. Therefore, we use the cosine transform. 
		\item Note that $\pd{y}$ is only every applied to $\eta_i$'s and $n_i$'s. 
		\begin{itemize}
			\item $\pd{y}\eta_i$'s occur only in the RHS of $\pd{t}n_i$. Since $\hat{\eta}_{i, k_x, k_y}$ are cosines and $\hat{n}_{i, k_x, k_y}$ are sines, this works out. $\pd{y}\rightarrow -k_y$.
			\item $\pd{y}n_i$'s occur only in the RHS of $\pd{t}\eta_i$. Since $\hat{\eta}_{i, k_x, k_y}$ are cosines and $\hat{n}_{i, k_x, k_y}$ are sines, this works out. $\pd{y}\rightarrow k_y$.
		\end{itemize}
	\end{itemize}
\end{itemize}
For each $k_x$ and $k_y$ pair, we get the following system.
\[\pd{t}\vec{x} = \hat{L}_{k_x,k_y}\vec{x}=
\left[\begin{array}{ccc|ccc}
0 & 0 & -\frac{1}{Fr^2}k_xi & 0 & 0 & -\frac{1}{Fr^2}k_xi \\
0 & 0 & \frac{1}{Fr^2}k_y & 0 & 0 & \frac{1}{Fr^2}k_y\\
-k_xi & -k_y & 0 & 0 & 0 & 0 \\ \hline
0 & 0 & -\frac{1}{Fr^2}k_xi & 0 & 0 &  -\frac{\alpha}{Fr^2}k_xi \\
0 & 0 & \frac{1}{Fr^2}k_y & 0  & 0&  \frac{\alpha}{Fr^2}k_y \\
0 & 0 & 0 & -k_xi & -k_y & 0
\end{array}\right] \begin{bmatrix}
\hat{m}_{1,k_x,k_y}\\
\hat{n}_{1,k_x,k_y}\\
\hat{\eta}_{1,k_x,k_y}\\
\hat{m}_{2,k_x,k_y}\\
\hat{n}_{2,k_x,k_y}\\
\hat{\eta}_{2,k_x,k_y}
\end{bmatrix} = \begin{bmatrix}
\pd{t}\hat{m}_{1,k_x,k_y}\\
\pd{t}\hat{n}_{1,k_x,k_y}\\
\pd{t}\hat{\eta}_{1,k_x,k_y}\\
\pd{t}\hat{m}_{2,k_x,k_y}\\
\pd{t}\hat{n}_{2,k_x,k_y}\\
\pd{t}\hat{\eta}_{2,k_x,k_y}
\end{bmatrix}
\]

Define an operator $\hat{L}$ such that it is a block diagonal matrix with all $(k_x, k_y)$ pairs of $\hat{L}_{k_x,k_y}$'s.
That is, we have:
\[
\hat{L} = \begin{bmatrix}
\hat{L}_{1,1} & 0 & 0 & 0 & 0 & \cdots & 0 \\
0 & \ddots & 0 & 0 & 0 & \cdots& 0 \\
0 & 0 & \hat{L}_{1, N_y}  & 0 & 0 & \cdots & 0\\
0 & 0 & 0 & \hat{L}_{2,1}  & 0 & \cdots & 0\\
0 & 0 & 0 & 0 & \ddots  & \vdots& 0\\
0 & 0 & 0 & 0 & 0 & \cdots& \hat{L}_{N_x,N_y}
\end{bmatrix}\]

Let $T$ and $T^{-1}$ represent DFT in $x-$ and DCT/DST in $y-$directions.
We need to solve
\begin{align*}
(I-h L)x^{(n+1)} &= x^{(n)} + h N(x^{(n)})\\
T(I-h L)x^{(n+1)} &= T\left[x^{(n)} + h N(x^{(n)})\right]\\
T(I-h L)T^{-1}T x^{(n+1)} &= T\left[x^{(n)} + h N(x^{(n)})\right]\\
(I-h TLT^{-1})\hat{x}^{(n+1)} &= \hat{x}^{(n)} + h TN(x^{(n)})\\
(I-h \hat{L})\hat{x}^{(n+1)}  &=\hat{x}^{(n)} + h TN(x^{(n)})
\end{align*}
This leads to solving blocks of all wavenumber pairs simultaneously.
$$
(I-h \hat{L}_{k_x,k_y})\hat{x}_{k_x,k_y}^{(n+1)}  =\hat{x}_{k_x,k_y}^{(n)} + h TN(x^{(n)})
$$

We manually compute the LU decomposition by using Gaussian elimination. 
\[A_{kx, ky} = I -  h \hat{L}_{k_x, k_y}= \begin{bmatrix}
1 & 0 & h\frac{1}{Fr^2}k_xi & 0 & 0 & h\frac{1}{Fr^2}k_xi \\
0 & 1 & -h\frac{1}{Fr^2}k_y & 0 & 0 & -h\frac{1}{Fr^2}k_y\\
hk_xi & hk_y & 1 & 0 & 0 & 0 \\ 
0 & 0 & h\frac{1}{Fr^2}k_xi & 1 & 0 &  h\frac{\alpha}{Fr^2}k_xi \\
0 & 0 & -h\frac{1}{Fr^2}k_y & 0  & 1&  -h\frac{\alpha}{Fr^2}k_y \\
0 & 0 & 0 & hk_xi & hk_y & 1
\end{bmatrix}\]
The following steps transform $A$ into an upper triangular form. It essentially performs $U = L^{-1}A$.

The main procedure is as follows:
$LUx = Ax = b$, where $b$ denotes whatever RHS we have.
Leting $y = Ux$ gives us $Ly = b$. Solve for $y$ then solve for $x$.
Apply $L^{-1}$ to $b$, Then, we are left with $Ux = L^{-1}b$.
Use backward substitution to solve for $x$.

 Let $R_i$ denote the $i^{th}$ row of $A$. Let $a_{k_x, k_y}:= 1+\frac{h^2}{Fr^2}(k_x^2+k_y^2)$, and let $b_{k_x, k_y}:= (a_{k_x, k_y}-1)(\alpha - 1)+\alpha$
\begin{enumerate}
	\item $R_3 \gets \frac{1}{a_{k_x, k_y}} \left(R_3 -h \left[i k_x R_1 + k_y R_2\right]\right)$
	\item $R_4 \gets R_4 - i \frac{h}{Fr^2}k_x R_3$
	\item $R_5 \gets R_5 + \frac{h}{Fr^2}k_y R_3$
	\item $R_6 \gets R_6 -h \left[ik_x R_4 +k_y R_5\right]$
\end{enumerate}
The resulting matrix is: \[U = L^{-1}A = 
\begin{bmatrix}
1 & 0 & i \frac{h}{Fr^2}k_x & 0 & 0 & i \frac{h}{Fr^2}k_x \\
0 & 1 & -\frac{h}{Fr^2}k_y & 0 & 0 & -\frac{h}{Fr^2}k_y \\
0 & 0 & 1 & 0 & 0 & 1 -\frac{1}{a_{k_x, k_y}} \\ 
0 & 0 & 0 & 1 & 0 & i \frac{h}{Fr^2}k_x \frac{b_{k_x,k_y}}{a_{k_x, k_y}} \\
0 & 0 & 0 & 0 & 1 & -\frac{h}{Fr^2}k_y \frac{b_{k_x, k_y}}{a_{k_x, k_y}} \\
0 & 0 & 0 & 0 & 0 & \frac{1+(a_{k_x, k_y}-1)(1+b_{k_x, k_y})}{a_{k_x, k_y}}
\end{bmatrix}\] 
The total set of operations is given below.
\begin{enumerate}
	\item $\mbfb{y_1}\gets \mbfr{b_1}$ (No need to store this separately.)
	\item $\mbfb{y_2}\gets \mbfr{b_2}$ (No need to store this separately.)
	\item $\mbfb{y_3} \gets \frac{1}{a_{k_x, k_y}} \left(\mbfr{b_3} -h \left[i k_x \mbfb{y_1} + k_y \mbfb{y_2}\right]\right)$ \\ 
	$\mbfb{y_3} \gets \frac{1}{a_{k_x, k_y}} \left(\mbfr{b_3} -h \left[i k_x \mbfr{b_1} + k_y \mbfr{b_2}\right]\right)$ 
	\item $\mbfb{y_4} \gets \mbfr{b_4} - i \frac{h}{Fr^2}k_x \mbfb{y_3}$
	\item $\mbfb{y_5} \gets \mbfr{b_5} + \frac{h}{Fr^2}k_y \mbfb{y_3}$
	\item $\mbfb{y_6} \gets \mbfr{b_6} -h \left[ik_x \mbfb{y_4} +k_y\mbfb{y_5}\right]$
	\item $\mbfg{x_6} \gets \frac{a_{k_x, k_y}}{1+(a_{k_x, k_y}-1)(1+b_{k_x, k_y})}\mbfb{y_6}$
	\item $\mbfg{x_5} \gets \mbfb{y_5} +\frac{h}{Fr^2}k_y \frac{b_{k_x,k_y}}{a_{k_x,k_y}}\mbfg{x_6}$
	\item $\mbfg{x_4} \gets \mbfb{y_4} - i \frac{h}{Fr^2}k_x\frac{b_{k_x,k_y}}{a_{k_x,k_y}}\mbfg{x_6}$
	\item $\mbfg{x_3} \gets \mbfb{y_3} - \left(1-\frac{1}{a_{k_x,k_y}}\right)\mbfg{x_6}$
	\item $\mbfg{x_2}\gets \mbfb{y_2} + \frac{h}{Fr^2}k_y(\mbfg{x_3}+\mbfg{x_6})$
	\item $\mbfg{x_1} \gets \mbfb{y_1} - i \frac{h}{Fr^2}k_x(\mbfg{x_3}+\mbfg{x_6})$
\end{enumerate}

Simplified:

\begin{enumerate}
	\item $\mbfb{y_3}\gets \frac{1}{a_{k_x, k_y}} \left(\mbfr{b_3} -h \left[i k_x \mbfr{b_1} + k_y \mbfr{b_2}\right]\right)$ 
	\item $\mbfb{y_4} \gets \mbfr{b_4} - i \frac{h}{Fr^2}k_x \mbfb{y_3}$
	\item $\mbfb{y_5} \gets \mbfr{b_5} + \frac{h}{Fr^2}k_y \mbfb{y_3}$
	\item $\mbfb{y_6} \gets \mbfr{b_6} -h \left[ik_x \mbfb{y_4} +k_y\mbfb{y_5}\right]$
	\item $\mbfg{x_6} \gets \frac{a_{k_x, k_y}}{1+(a_{k_x, k_y}-1)(1+b_{k_x, k_y})}\mbfb{y_6}$
	\item $\mbfg{x_5} \gets \mbfb{y_5} +\frac{h}{Fr^2}k_y \frac{b_{k_x,k_y}}{a_{k_x,k_y}}\mbfg{x_6}$
	\item $\mbfg{x_4} \gets \mbfb{y_4} - i \frac{h}{Fr^2}k_x\frac{b_{k_x,k_y}}{a_{k_x,k_y}}\mbfg{x_6}$
	\item $\mbfg{x_3} \gets \mbfb{y_3} - \left(1-\frac{1}{a_{k_x,k_y}}\right)\mbfg{x_6}$
	\item $\mbfg{x_2}\gets \mbfr{b_2} + \frac{h}{Fr^2}k_y(\mbfg{x_3}+\mbfg{x_6})$
	\item $\mbfg{x_1} \gets \mbfr{b_1} - i \frac{h}{Fr^2}k_x(\mbfg{x_3}+\mbfg{x_6})$
\end{enumerate}

ALTERNATIVELY:

 Let $R_i$ denote the $i^{th}$ row of $A$. Let $a_{k_x, k_y}:= \frac{1}{1+\frac{h^2}{Fr^2}(k_x^2+k_y^2)}$, and let $b_{k_x, k_y}:= (\frac{1}{a_{k_x, k_y}}-1)(\alpha - 1)+\alpha$
\begin{enumerate}
	\item $R_3 \gets a_{k_x, k_y} \left(R_3 -h \left[i k_x R_1 + k_y R_2\right]\right)$
	\item $R_4 \gets R_4 - i \frac{h}{Fr^2}k_x R_3$
	\item $R_5 \gets R_5 + \frac{h}{Fr^2}k_y R_3$
	\item $R_6 \gets R_6 -h \left[ik_x R_4 +k_y R_5\right]$
\end{enumerate}
The resulting matrix is: \[U = L^{-1}A = 
\begin{bmatrix}
1 & 0 & i \frac{h}{Fr^2}k_x & 0 & 0 & i \frac{h}{Fr^2}k_x \\
0 & 1 & -\frac{h}{Fr^2}k_y & 0 & 0 & -\frac{h}{Fr^2}k_y \\
0 & 0 & 1 & 0 & 0 & 1- a_{k_x, k_y} \\ 
0 & 0 & 0 & 1 & 0 & i \frac{h}{Fr^2}k_x a_{k_x, k_y}b_{k_x,k_y} \\
0 & 0 & 0 & 0 & 1 & -\frac{h}{Fr^2}k_y  a_{k_x, k_y}b_{k_x,k_y}\\
0 & 0 & 0 & 0 & 0 & a_{k_x, k_y}[1+(\frac{1}{a_{k_x, k_y}}-1)(1+b_{k_x, k_y})]
\end{bmatrix}\] 
The total set of operations is given below.

\begin{enumerate}
	\item $\mbfb{y_3}\gets a_{k_x, k_y} \left(\mbfr{b_3} -h \left[i k_x \mbfr{b_1} + k_y \mbfr{b_2}\right]\right)$ 
	\item $\mbfb{y_4} \gets \mbfr{b_4} - i \frac{h}{Fr^2}k_x \mbfb{y_3}$
	\item $\mbfb{y_5} \gets \mbfr{b_5} + \frac{h}{Fr^2}k_y \mbfb{y_3}$
	\item $\mbfb{y_6} \gets \mbfr{b_6} -h \left[ik_x \mbfb{y_4} +k_y\mbfb{y_5}\right]$
	\item $\mbfg{x_6} \gets \frac{y_6}{a_{k_x, k_y}[1+(\frac{1}{a_{k_x, k_y}}-1)(1+b_{k_x, k_y})]}$
	\item $\mbfg{x_5} \gets \mbfb{y_5} +\frac{h}{Fr^2}k_y a_{k_x, k_y}b_{k_x,k_y}\mbfg{x_6}$
	\item $\mbfg{x_4} \gets \mbfb{y_4} - i \frac{h}{Fr^2}k_xa_{k_x, k_y}b_{k_x,k_y}\mbfg{x_6}$
	\item $\mbfg{x_3} \gets \mbfb{y_3} - \left(1-a_{k_x,k_y}\right)\mbfg{x_6}$
	\item $\mbfg{x_2}\gets \mbfr{b_2} + \frac{h}{Fr^2}k_y(\mbfg{x_3}+\mbfg{x_6})$
	\item $\mbfg{x_1} \gets \mbfr{b_1} - i \frac{h}{Fr^2}k_x(\mbfg{x_3}+\mbfg{x_6})$
\end{enumerate}
Since $(i \frac{h}{Fr^2} k_x, \frac{h}{Fr^2}k_y)$ appear more often than $(kx, ky)$, we will store those values for coding purposes. 

\end{document}